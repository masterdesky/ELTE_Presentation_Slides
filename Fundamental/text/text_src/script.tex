\section{Intro}
I was given to process chapter 12 and chapter 13 from Smolin's book. First of the chapters is titled as \q{\textit{Quantum Mechanics and the Liberation of the Atom}}, while the second one is \q{\textit{The Battle Between Relativity and the Quantum}}.

\section{Summary}
Just for some clarification and to make it easier to understand and comprehend what I'm going to talk about, I wanted to share some of my insights/remarks on the idea proposed in Smolin's book in general and in these two chapters. But first I would like to give a summary and explanation of this - according to Smolin - \q{revolutionary idea}, and how these chapters are connected to it. \newline
So, in the book Smolin speaks about a lot of things in particular -- lot of physical theories and methods --, and explains why does he think they're incorrect. But there is a single, fundamental idea, which all these topics are wrapped around. And it is that \q{\textit{Time is real}} as he says, and he continuously repeats this short phrase throughout the whole book. But what does he mean by this? \newline
Smolin declares, that science's current view on time itself and how science handles it, is incorrect. Some \q{background infos}: the concept of time in modern physics originates from Einstein's theory of relativity. In this framework space and time is merged into one entity which is called as \q{spacetime}. Like this, time simply ceases to exist as an individual quantity, but rather becomes a coordinate, just like the coordinates of space itself in geometry. Of course, this could be rephrased negatively as \q{time gets downgraded}, and \q{space becomes fundamental} or \q{space becomes real}, while \q{time ceases to exist as real}, because the behaviour of time is now simply defined as space's does. This is the concept, which Smolin criticizes and which - according to Smolin - makes us a lot of troubles in physics. He proposes we should make it the other way around. To consider time as \q{real} and space maybe just as an other aspect of it. \par
In Chapter 12 he discusses QM. He argues, that it isn't the correct theory and it should be succeeded by something else, which is supposedly consistent with his concept of \q{real time}. In chapter 13 blabla.

\section{My own insights}
I must confess, I love these kind of books and writings, where the author tries to answer fundamental questions by analysing the bigger picture and the small details simultaneously. Another popular contemporary author, Yuval N. Harari writes in seemingly the same manner, as Smolin does -- or at least this book was reminiscent of his style. But I had the same problem with both authors, which strongly influenced me and also influences me currently in the manner of my discussion about Smolin's work during this presentation. (That's the only reason why I'm mentioning these thoughts.) I have a strong feeling, that first of all, both of these authors try to convey/broadcast a narrative to the audience. But while they're doing so, they also try to give the appearance of having a powerful argument supporting their conveyed narrative. Doing this by actually cherry picking small details from pretty obscure and unclear topics, declaring, that \q{this unknown something} implies the rightness of their idea. I'm literally no one to criticize anyhow the scientific work of Lee Smolin a researcher of loop quantum-gravity, but this kind of discussion tasted unpleasantly \q{pseudoscience-y} to me. That's why both Smolin and Harari fails to convince me about, that their ideas are outstanding somehow. \par
Of course we can ask ourselves the question: does it actually matter? I think it's not just simply appropriate, I think it's significantly important to share different narratives, thoughts and ideas with the world. We can learn anything about others and ourselves -- as humans -- by listening to these ideas, and we can never be sure, when one of this idea could become the new theory of relativity. If that's the only goal of this book, then I'm impressed, it's a great work actually. But it is not so great to claim anything more on behalf of an idea, what it actually has to offer. Smolin actually doesn't claim that his theory is correct -- since no experiment supports it -- but he claims that it could revolutionize our perspective on a lot of things, eg. physics, economics, social problems, etc... And I just can't see anything, which supports this pretty monumental claim. \par
That was my thoughts, now move onto discussing the chapters.


\section{Chapter 12}
This chapter completely revolves around the so-called \textit{principle of precedence}. Smolin first enlightens the reader why he thinks, that QM is not a final theory, while pointing out some fundamental flaws in the scientific method of the QM itself. (I'll tell you in a minute, what are these exactly.) He also draws parallel between two pair of theories: Newton's theory of gravitation and its successor, the theory of general relativity, and quantum mechanics and its hypothetical successor, some improved, \q{final} theory. He then details, that this theory should use the principle of precedence, which in a philosophical sense, incorporates his \q{time is real} theory. \par
So, what problems Smolin found in QM? He enumerates exactly three of them:
\begin{enumerate}
	\item Its failure to give a physical picture of our world. (That's true, QM is mostly abstract math with a lot of uninterpretable notions or incomprehensible explanations.)
	\item Its failure to predict the exact outcome of experiments. (That's also true, QM gives us only probabilities about the things which could happen.)
	\item Its failure to incorporate the measuring instruments, observers, etc. into the theory. (If our goal is to create a cosmological/final model, that's of course a serious problem. It's supposed to describe everything.)
\end{enumerate}
After this, Smolin describes three well-known problems in QM, which still lacks of explanation. These are the problem of noncommuting variables (why we can't measure the position and momentum of a particle at the same time?), the entanglement (what does happen, when two particles get entangled?) and nonlocality (how does two, entangled particle act on each other instantaneously, even if they're far away?). I won't speak more about this. \par
To resolve this problem, Smolin proposes, what if our world is a subject to the principle of precedence, instead of timeless laws? The difference between these two is, that according to the principle of precedence, the outcome of experiments, or events are determined by the outcomes of the same particular events in the past. However according the the \q{principle of timeless laws}, all physical laws are created at the Big Bang, and thus the outcome of every unprecedented event could be predicted simply by knowing these laws. If the principle of precedence is true, it means, that in every experiment, where an unprecedented event about to happen, the particles are \q{free to choose} the outcome and we can't do anything to predict that. That's the liberation of the atom as stands in the title. \par
According to Smolin, QM also follows the \textit{principle of maximal freedom}. A particle tries to maximize its freedom (the number of informations, which is needed to predict anything about its properties).